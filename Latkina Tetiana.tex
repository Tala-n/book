\documentclass[twoside]{book}
% A6: 1.03, 2.305, 1.33, 1.73
\usepackage[top=1.83cm, bottom=2.105cm, outer=2.13cm, inner=2.53cm, headsep=7pt, a5paper]{geometry}
\usepackage{amsmath,amsthm,amssymb}
\usepackage{mathtext}
\usepackage[T1,T2A]{fontenc}
\usepackage[utf8]{inputenc}
\usepackage[english,bulgarian,ukrainian]{babel}




%Bibliography
\usepackage{natbib}

% Custom section/chapter titles
\usepackage[explicit]{titlesec}

% Drawing
\usepackage{graphics}
\usepackage{tikz}

% Fonts
\usepackage{fontspec}
\usepackage{textalpha}
\usepackage{lettrine}
\usepackage{GoudyIn}
\usepackage{needspace}
\usepackage{textcase}

%custom header/footer
\usepackage{fancyhdr}
\pagestyle{fancy}


% Multi column
\usepackage{multicol}
\usepackage{multicolrule}
\setlength{\columnsep}{15pt}

% Table of contents
\usepackage[hidelinks]{hyperref}
\hypersetup{
	colorlinks=false,
	linktoc=all,
}

\usepackage{pbox}
%\usepackage[activate={true,nocompatibility}]{microtype}


\title{Привіт}
\author{Це моя перша робота}
\date{Грудень 2022}

\graphicspath{{images-overview/}{images-deconv/}{images-linearization/}{images-noiseincoding/}{images-linearization/imagemotion/}{images-opticalcoding/}}

\begin{document}
	\maketitle

	\documentclass{
		У світі багато фортифікаційних споруд. Тільки в Україні їх понад 100 й серед них –  зруйнований замок Вишневецьких  в Лубнах‚ споруджений згідно з передовими ідеями европейського фортифікаційного мистецтва.  Але й він не витримав осади селянсько-козацького війська під час визвольної війни українського народу.Пройшов час і його руїнами зацікавились вчені. Перші свідоцтва про підземелля були зібрані ще наприкінці ХІХ ст. Створення під Лубнами другого підземного міста пов’язують з українським магнатом, князем Єремією Вишнівецьким. Він володів 53 містами і містечками, багатьма селами і численними замками. У 1640 році володар Вишнівеччини перетворив Лубни на свою столицю. Для будівництва замка-фортеці були запрошені досвідчені архітектори, монахи Бенедиктинського ордену. За рік на високому пагорбі  над річкою Супій був споруджений неприступний замок з високими стінами й вежами. Згідно з леґендою замок славився неабиякою розкішшю. Тільки посуду з срібла в замку зберігалось декілька возів. Князь заховав коштовності в підземеллях міста. Коли козаки взяли штурмом місто і повністю зруйнували його, скарбів не знайшли. За наказом князя інженери збудували тунелі, що залягали на глибині від трьох до семи метрів. Ці тунелі мають ґотичні склепіння і часто-густо являють заплутані лябіринти з багаточисельними відгалужженями, які невідомо куди йдуть. Археологічні розкопки, які проводив археолог з Москви Гнат Стеллецький‚ вимушені були припинити через початок Першої світової війни. Складний рельєф місцевости та функціональні вимоги зумовили розплянування міста на три взаємопов’язані частини – замок, цитадель, передзамкове укріплення й передмістя. У ХVІІ ст. захисні огорожі укріплення складалися із земляних валів, ровів, 15 дерев’яних башт, палісадів та рогаток. З часом ці укріплення занепали і на середину ХVІІІ ст. від них залишились переважно вали, рови та поодинокі вежі над фортечними брамами. За свідченням пляну  Лубен 1748 року до фортеці вело двоє воріт і чотири фіртки, а на її території знаходилось три храми, будинки полкової та сотенної канцелярій, школи, шпиталі, торгові ряди, двори старшин, козацькі та міщанські оселі. Загальна довжина фортечних укріплень сягала 1,700 сажень. Дотепер на місцевості простежуються залишки земляних укріплень, а як виглядали вежі можна побачити на старовинній фотографії. В наш час, коли Україна відроджується‚ настав час віднайти гроші на дослідження замку в Лубнах і провести часткову реконструкцію укріплень і підземель.
	}

	
\end{document}